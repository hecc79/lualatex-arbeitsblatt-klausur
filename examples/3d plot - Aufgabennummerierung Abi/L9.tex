%!TEX root = 2024-Q2M05-Mathematik-2-HecC.tex
\kloesung

\begin{soltable}{}
\textbf{Landesabitur Hessen 2020 GK Teil C1} & \\ \hline 
 $A(20|20|0), C(–20|–20|0)$ & 2 \\ \hline
 $V=\frac{1}{3}\cdot 40^2 \cdot 30 = 16000 [m^3]$ & 2 \\ \hline
 $L=4\cdot \left|\overrightarrow{DS}\right|=4\cdot \sqrt{20^2+20^2+30^2}=4\sqrt{1700}≈164,92[m]$ & 3 \\ \hline 
 $\cos(\alpha)=\vectcos{\overrightarrow{SD}}{\overrightarrow{SA}}=\vectcos{\V{20}{-20}{-30}}{\V{20}{20}{-30}}=\dfrac{900}{1700}=\dfrac{9}{17}\Rightarrow \alpha=\ang{58}$ & 3 \\ \hline 
 $g=2\cdot 4 \cdot \sin\left(\dfrac{\ang{58}}{2}\right)≈3,88 [m]$ & 1 \\ 
 $h=4\cdot\cos\left( \frac{\ang{58}}{2}\right)≈3,50 [m]$ & 1 \\ Die Punkte M und M' stimmen in den y- und z-Koordinaten überein. Daher liegen sie
auf einer Geraden, die parallel zur x-Achse verläuft. Somit ist M' der Punkt, in dem
der Dachfirst des Vorbaus auf die Pyramide trifft.
 $A=4\cdot\frac{1}{2}\cdot a\cdot h≈27 [m^2]$ & 2 \\ \hline
 $E_{ASD}: \EbeneP{20}{20}{0}{-20}{-20}{30}{0}{-40}{0}$ & 2 \\
 mit Weg: $\overrightarrow{n}=\V{3}{0}{2}$ & 2 \\
 mit Weg: $E_{ASD}: 3x+2z=60$ & 2 \\ \hline 
 $M' \text{ in } E_{ASD}:~ e\cdot10+2\cdot 15=60\Rightarrow M' \in E_{ASD}$ & 1 \\ 
Die Punkte $M$ und $M'$ stimmen in den y- und z-Koordinaten überein. Daher liegen sie
auf einer Geraden, die parallel zur x-Achse verläuft. Somit ist $M'$ der Punkt, in dem
der Dachfirst des Vorbaus auf die Pyramide trifft. & 2 \\ 
\begin{tikzpicture}[
]
\begin{axis}[schule3,
    xmin=-25, xmax=49,
    ymin=-35, ymax=35,
    zmin=0, zmax=34,
    scale=0.1,
    axis line style={line width=0.5pt, -{Stealth[scale=1.2]}},
    extra x ticks={10,20,30,40},
    %extra x tick labels={10,20,30,40},
    extra y ticks={-30,-20,-10,10,20,30},
    %extra y tick labels={-30,-20,-10,10,20,30},
    %extra z ticks={-5,1,5},
    %extra z tick labels={-5,1,5},
    xtick distance=10, ytick distance=10, ztick distance=10,
    font=\normalsize,
    ]
\begin{pgfonlayer}{axis background}
    \begin{scope}[canvas is yz plane at x=0]
		\draw[step=5,gray] (-35,-25) grid (35,35);
    \end{scope}
\end{pgfonlayer}


	\coordinate (A) at (20,20,0);
	\coordinate (A2) at (2,2,27);
	\coordinate (B) at (-20,20,0);
	\coordinate (B2) at (-2,2,27);
	\coordinate (C) at (-20,-20,0);
	\coordinate (D) at (20,-20,0);
	\coordinate (D2) at (2,-2,27);
	\coordinate (S) at (0,0,30);
	\coordinate (M) at (40,0,15);
	\coordinate (H) at (40,5,12.5);
	\coordinate (M1) at (10,0,15);
	\coordinate (F) at (40,5,0);
	\coordinate (F1) at (20,5,0);
	\coordinate (E) at (40,-5,0);
	\coordinate (E1) at (20,-5,0);
	\coordinate (I) at (40,-5,12.5);
	\coordinate (I1) at (35/3,-5,12.5);
	\coordinate (H1) at (35/3,5,12.5);

	\draw[thick, fill=lightgray] (D2)--(A2)--(B2)--(S)--(D2);
	
	\draw[thick] (A)--(B);
	\draw[thick,dashed] (B)--(C)--(D);
	\draw[thick] (A)--(S);
	\draw[thick] (B)--(S);
	\draw[thick,dashed] (C)--(S);
	\draw[thick] (D)--(S);

	%Vorbau
	\draw[thick] (E)--(F)--(H)--(M)--(I)--cycle;	
	\draw[thick] (H)--(I);	
	\draw[thick] (F)--(F1);
	\draw[thick] (H)--(H1);
	\draw[thick] (M)--(M1);
	\draw[thick] (I)--(I1);
	\draw[thick] (I1)--(M1)--(H1)--(F1);
	\draw[thick,dashed] (E)--(E1)--(I1);
	\draw[thick] (A)--(F1);
	\draw[thick] (D)--(20,-15,0);
	\draw[thick,dashed] (F1)--(20,-15,0);

	%Beschriftung
	\draw (M1) node[above]{$\mathbf{M'}$};

\end{axis}
\end{tikzpicture} & 1 \\ \hline
Geradengleichung der Dachkante durch Punkt H & \\ 
$h: \Gerade{40}{5}{12,5}{-1}{0}{0}$ & 2 \\
$h\cap E_{ASD}:~ 3(40-r)+2\cdot 12,5=60 \Leftrightarrow -3r+145=60 \Leftrightarrow r=\dfrac{85}{3}$ & 1 \\ 
In Gerade einsetzen: $H'\left(\dfrac{35}{3} ~\bigg| ~ 5 ~ \bigg| ~ 12,5  \right)$ & 1 \\ \hline 
Da die beiden Kanten $\overrightarrow{MM'}$ und $\overrightarrow{HH'}$ parallel zur x-Achse und somit parallel zueinander verlaufen und zudem $\left| \overrightarrow{HH'} \right|=\dfrac{85}{3}\neq 30 = \left|\overrightarrow{MM'}\right|$, handelt es sich bei dem Viereck $HH'M'M$ um ein Trapez. & 2 \\ 
$\left|\overrightarrow{MH}\right|=\left| \V{0}{5}{-2,5}\right|= \sqrt{31,25}\approx 5,59 [m]$ & 1 \\ 
$A=\frac{1}{2}\cdot\left(\left|\overrightarrow{MM'
}\right|+ \left|\overrightarrow{HH'
}\right|  \right)\cdot \left|\overrightarrow{MH
}\right|\approx 163,05 [m^2]$ & 3 \\ \hline 
In Zeile (1) wird eine Gleichung der Gerade
$g_{MS}$ durch die Punkte M und S angegeben. & 1 \\ 
In Zeile (2) wird der Wert für den Geradenparameter berechnet, bei dem die z-Koordinate der Gerade den Wert 1,6 hat.& 2\\ 
In Zeile (3) wird durch Einsetzen des in Zeile (2) berechneten Parameterwertes in die
Geradengleichung $g_{MS}$ der Ortsvektor des Punktes bestimmt, der auf der Geraden liegt und dessen z-Koordinate den Wert 1,6 besitzt. & 1 \\ 
Der Punkt P gibt die Position des Besuchers auf dem Boden an, ab der er die Pyrami-
denspitze S nicht mehr sehen kann, wenn er sich der Pyramide entlang der x-Achse
nähert, weil die Pyramidenspitze durch den Vorbau verdeckt wird. & 2
\end{soltable}
