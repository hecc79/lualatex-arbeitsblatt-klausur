%!TEX root = 2024-Q2M05-Mathematik-2-HecC.tex
%LA 2020 GK HT C1
\teil{40}

\begin{abienum}
	\item Die „Pyramide Mainz“ (Material 1) ist eine Lokalität, die man für Veranstaltungen mieten kann.  Material 2 zeigt einen Planungsentwurf für ein ähnliches Gebäude, das aus einer quadratischen Pyramide mit einer Grundseite der Länge 40 m und einem parallel zur x-Achse ausgerichteten Vorbau mit rechteckiger Grundfläche und symmetrischem Dach besteht. Der Ursprung des Koordinatensystems liegt in der Mitte der Grundfläche der Pyramide. Es sind der Punkt $D(20|-20|0)$ der Grundfläche und die Spitze $S(0|0|30)$ der Pyramide gegeben. Alle Einheiten sind in Meter angegeben.

	\begin{abienum}
		\item Geben Sie die Koordinaten der Punkte A und C an.~\hfill\textbf{(2BE)}
		\item Berechnen Sie das Volumen der Pyramide. ~\hfill\textbf{(2BE)}
		\item Entlang der vier Seitenkanten der Pyramide werden Lichterketten angebracht. \\
		Berechnen Sie die Gesamtlänge der Lichterketten. ~\hfill\textbf{(3BE)}
		\item  Berechnen Sie den Winkel an der Spitze eines Seitendreiecks der Pyramide. \\
		~[zur Kontrolle: $\measuredangle DSA ≈\ang{58}$] ~\hfill\textbf{(3BE)}
		\item Die Spitze der Pyramide ist mit Metall verkleidet. Die Seitenkanten dieser ebenfalls quadratischen Pyramide sind \SI{4}{m} lang (Material 2). \\
		Berechnen Sie die Größe der Fläche, die mit Metall verkleidet ist.~\hfill\textbf{(4BE)}
\item Die Seitenfläche der Pyramide mit den Eckpunkten A, S und D liegt in der Ebene $E_{ASD}$. 
Geben Sie eine Parameterform der Ebene $E_{ASD}$ an und bestimmen Sie eine zugehörige Koordinatengleichung. \\
~[zur Kontrolle: Eine mögliche Koordinatengleichung lautet $E_{ASD}: 3x + 2z = 60$.]~\hfill\textbf{(6BE)}
	\end{abienum}
	\item Im Folgenden wird das Dach des Vorbaus in Material 2 betrachtet. Von der Vorderseite dieses 
Daches sind die Punkte $M(40|0|15)$ und $H(40|5|12,5)$ gegeben. 

\begin{abienum}
	\item Zeigen Sie, dass der Punkt $M'(10|0|15)$ auf der Ebene $E_{ASD}$ liegt, und begründen Sie ohne weitere Rechnung, dass es sich bei dem Punkt $M'$ um denjenigen Punkt handeln muss, in dem der (durch den Punkt M verlaufende) Dachfirst des Vorbaus auf die Pyramide trifft. Beschriften Sie den Punkt $M'$ in Material 2. ~\hfill\textbf{(4BE)}
	\item Berechnen Sie die Koordinaten des Punktes $H'$, bei dem die Dachkante des Vorbaus (Traufe), die durch den Punkt H verläuft, auf die Pyramide trifft. \\
	~[Kontrolle: $\Punkt{H'}{\frac{35}{3}}{5}{12,5}$]~\hfill\textbf{(4BE)}
	\item Untersuchen Sie, um welche Art von Viereck es sich bei der Dachfläche $HH'M'M$ handelt, und 
bestimmen Sie den Flächeninhalt der Dachfläche $HH'M'M$. ~\hfill\textbf{(6BE)}
\end{abienum}
\newpage
\item Ein Besucher nähert sich dem Pyramideneingang entlang der x-Achse aus positiver Richtung. Die Augenhöhe des Besuchers ist \SI{1,60}{m} über dem Boden. Erläutern Sie die Rechnung in den Zeilen (1) bis (3) im untenstehenden Kasten und erklären Sie die Bedeutung des Punktes P aus 
Zeile (4) im Sachzusammenhang. ~\hfill\textbf{(6BE)}
\end{abienum}
\begin{mdframed}
\vspace{-0.5cm}
	\begin{gather}
		\Punkt{M}{40}{0}{15}, \Punkt{S}{0}{0}{30} \Rightarrow g_{MS}: \GeradeP{40}{0}{15}{-40}{0}{15}{r}\\
		1,6=15+r\cdot15 \Leftrightarrow r=-\frac{67}{75}\\
		\overrightarrow{x}=\V{40}{0}{15}-\frac{67}{75}\cdot \V{-40}{0}{15}=\V{\frac{1136}{15}}{0}{\frac{8}{5}}≈ \V{75,73}{0}{1,6}\\
		P(75,73|0|0)		
	\end{gather}	
\end{mdframed}
\newpage
\textsf{\textbf{Material 1: }Die „Pyramide Mainz“ in Mainz-Hechtsheim}


https://www.pyramidemainz.de/die-pyramide (abgerufen am 11.06.2019). 

\textsf{\textbf{Material 2:}}

\begin{tikzpicture}[
]
\begin{axis}[schule3,
    xmin=-25, xmax=49,
    ymin=-35, ymax=35,
    zmin=0, zmax=34,
    scale=0.2,
    axis line style={line width=0.5pt, -{Stealth[scale=1.2]}},
    extra x ticks={10,20,30,40},
    extra x tick labels={10,20,30,40},
    extra y ticks={-30,-20,-10,10,20,30},
    extra y tick labels={-30,-20,-10,10,20,30},
    %extra z ticks={-5,1,5},
    %extra z tick labels={-5,1,5},
    xtick distance=10, ytick distance=10, ztick distance=10,
    font=\normalsize,
    ]
\begin{pgfonlayer}{axis background}
    \begin{scope}[canvas is yz plane at x=0]
		\draw[step=5,gray] (-35,-25) grid (35,35);
    \end{scope}
\end{pgfonlayer}


	\coordinate (A) at (20,20,0);
	\coordinate (A2) at (2,2,27);
	\coordinate (B) at (-20,20,0);
	\coordinate (B2) at (-2,2,27);
	\coordinate (C) at (-20,-20,0);
	\coordinate (D) at (20,-20,0);
	\coordinate (D2) at (2,-2,27);
	\coordinate (S) at (0,0,30);
	\coordinate (M) at (40,0,15);
	\coordinate (H) at (40,5,12.5);
	\coordinate (M1) at (10,0,15);
	\coordinate (F) at (40,5,0);
	\coordinate (F1) at (20,5,0);
	\coordinate (E) at (40,-5,0);
	\coordinate (E1) at (20,-5,0);
	\coordinate (I) at (40,-5,12.5);
	\coordinate (I1) at (35/3,-5,12.5);
	\coordinate (H1) at (35/3,5,12.5);

	\draw[thick, fill=lightgray] (D2)--(A2)--(B2)--(S)--(D2);
	\draw[Stealth-Stealth,thick] (-1,0,30)--(-3,2,27) node[midway,above]{4};
	
	\draw[thick] (A)node[below]{A}--(B)node[right]{B};
	\draw[thick,dashed] (B)--(C)node[left]{C}--(D)node[below]{D};
	\draw[thick] (A)--(S);
	\draw[thick] (B)--(S);
	\draw[thick,dashed] (C)--(S);
	\draw[thick] (D)--(S)node[above left]{S};

	%Vorbau
	\draw[thick] (E)--(F)--(H)--(M)--(I)--cycle;	
	\draw[thick] (H)--(I);	
	\draw[thick] (F)--(F1);
	\draw[thick] (H)--(H1);
	\draw[thick] (M)--(M1);
	\draw[thick] (I)--(I1);
	\draw[thick] (I1)--(M1)--(H1)--(F1);
	\draw[thick,dashed] (E)--(E1)--(I1);
	\draw[thick] (A)--(F1);
	\draw[thick] (D)--(20,-15,0);
	\draw[thick,dashed] (F1)--(20,-15,0);

	%Beschriftung
	\draw (E) node[below left]{E};
	\draw (F) node[below right]{F};
	\draw (H) node[right]{H};
	\draw (M) node[below]{M};
	\draw (H1) node[above]{$H'$};

\end{axis}
\end{tikzpicture}

\textsf{\textbf{Material 3:}}

Quelle: \url{https://www.sanier.de/dach/fachbegriffe-rund-ums-dach} (abgerufen am 2025-05-31)
