\documentclass[11pt,parskip=half,headings=small,german]{klausur}
\begin{document}
\newcommand{\jahr}{2024}
\newcommand{\datum}{2025-05-23}
\newcommand{\lehrer}{}
\newcommand{\schule}{}
\newcommand{\kurs}{}
\newcommand{\fach}{Mathematik}
\newcommand{\thema}{Einf. i. d. Analysis, Exponentialfunktionen, Kettenregel}
\newcommand{\nr}{2}
\newcommand{\nrmax}{2}
\newcommand{\art}{Klausur}
\newcommand{\zeit}{90 min}
\newcommand{\be}{52}
\newcommand{\bes}{8}


\klausurkopf
\begin{tblr}{width=\linewidth, colspec={*{17}{X[c]}}, hline{2}, vline{2-17}}
15 & 14 & 13 & 12 & 11 & 10 & 09 & 08 & 07 & 06 & 05 & 04 & 03 & 02 & 01 & 00 & $\varnothing$   \\
~  & ~  & ~  & ~  & ~  & ~  & ~  & ~  & ~  & ~  & ~  & ~  & ~  & ~  & ~  & ~  & ~ \\
\end{tblr}
\mytitle

\textsf{\textbf{Zugelassene Hilfsmittel:}} WTR, Formelsammlung, Geodreieck

%!TEX root = 2024-E25-Mathematik-2-HecC.tex
\kaufgabe{10}

\begin{tikzpicture}[
]
\begin{axis}[schule,
	xlabel=x,
	ylabel=y,
	x=1cm,
	y=1cm,
	xtick distance=1,
	ytick distance=1,
	xmin=-4.5,
	xmax=5.5,
	ymin=-1.5,
	ymax=7.5
]
	\addplot [domain=-5:5.5] {1.5*2^x};
	\addplot [domain=-5:5.5] {2.0*0.75^x};
	\addplot [domain=-5:5.5,dashed] {2^x+1};
	\addplot [domain=-5:5.5] {2*0.5^x-1};

	\node[below left] (0) at (0,-2pt) {\scriptsize $0$};
	\node[above] at (-4,0.1){\scriptsize A};
	\node[above] at (-4,1.1){\scriptsize B};
	\node[below] at (5,-1){\scriptsize C};
	\node[above] at (5,0.5){\scriptsize D};

\end{axis}

\end{tikzpicture}

Geben Sie für die fünf Graphen A, B, C, D und E jeweils eine Funktionsgleichung an.
%!TEX root = 2025-G5d-Mathematik-2-Hec.tex
\kaufgabe{3}

Auf einer Rolle Geschenkpapier befinden sich noch \SI{3,30}{m}
Geschenkpapier. Zum Verpacken der Geschenke wird ein Stück der Länge
\SI{40}cm und drei Stücke der Länge \SI{60}{cm} benötigt. Das übrige
Geschenkpapier wird in zwei gleichgroße Reststücke aufgeteilt.

Berechne die Länge des Reststücks. 
%!TEX root = 2025-G5d-Mathematik-2-Hec.tex
\kaufgabe{}

Tom ist 14 cm größer als Frank, aber 5 cm kleiner als Sarah. Lena ist
1{,}45 m 	groß und 12 cm kleiner als Sarah. Wie groß sind die
Kinder? Gib alle Rechenwege an und schreibe eine Antwort.




%!TEX root = 2024-E25-Mathematik-2-HecC.tex
\kaufgabe{20}

In einem Labor wird die Wirksamkeit eines neuen Mittels gegen die Ausbreitung von Stechmücken untersucht.
Bei einem ersten Versuch beschreibt die Funktion $N(t)=150\cdot e^{\num{0,25}t}$ ($t$ in Tagen) modellhaft die Population von Stechmücken. $N(t)$ ist die Anzahl der Stechmücken zum Zeitpunkt $t$. Die Beobachtung beginnt bei $t=0$.

\begin{tasks}(1)
	\task Berechnen Sie die Populationsgröße zum Zeitpunkt 6 Tage nach Beobachtungsbeginn.
	\task Berechnen Sie den Zeitpunkt, zu dem die Populationsgröße die Anzahl von 1500 Stechmücken erreicht.
	\task Berechnen Sie die Verdopplungszeit.
	\task Erklären Sie, dass die Funktion $N(t)$ auch durch die Funktionsgleichung $N(t)=150\cdot \num{1,284}^t$ beschrieben werden kann. Erläutern Sie den Wert von \num{1,284} im Sachzusammenhang.
	\task Berechnen Sie die durchschnittliche Wachstumsgeschwindigkeit während der ersten 6 Tage.
	\task Berechnen Sie die momentane Wachstumsgeschwindigkeit am Ende des 6. Tages.
	\task Von einer anderen Mückenart werden am Ende des zweiten Tages 130 Mücken und am Ende des vierten Tages 300 Mücken gezählt. Berechnen Sie eine geeignete Funktionsgleichung, die dieses Wachstum als exponentielles Wachstum beschreibt.
\end{tasks}
%!TEX root = 2024-E25-Mathematik-2-HecC.tex
\kaufgabe{12}

Der Abiturjahrgang 2025 hat vor Weihnachten heißen Glühwein auf dem Weihnachtsmarkt verkauft, um die Abikasse aufzubessern. Der Glühwein wird auf \ang{70}C erhitzt. Wenn der Glühwein in eine Tasse gefüllt wird, kühlt dieser bei einer Umgebungstemperatur von \ang{0}C innerhalb von 10 Minuten um die Hälfte ab.

\begin{tasks}(1)
	\task Leiten Sie eine Funktionsgleichung $T(t)$ her, die die oben beschriebene Abkühlung von Glühwein beschreibt. $T(t)$ gibt hierbei die Temperatur in Grad Celsius und $t$ die Zeit in Minuten an. Runden Sie auf vier Nachkommastellen. 

	[Kontrollergebnis: $T(t)=70\cdot e^{-0,0693t}$]
	\task Bestimmen Sie die Temperatur des Glühweins nach 5 Minuten.
	\task Berechnen Sie, um wie viel Prozent der Glühwein nach 15 Minuten abgekühlt ist.
	\task Bei einer Temperatur von \ang{60}C lässt sich der Glühwein angenehm trinken. Berechnen Sie, wie lange ein Kunde warten muss, bis diese Temperatur erreicht ist.
	\task Diskutieren Sie, wann der Glühwein eine Temperatur von \ang{0}C erreicht hat.
\end{tasks}



\newpage
{\large \textbf{\textsf{Erwartungshorizont}}}
%!TEX root = 2024-E25-Mathematik-2-HecC.tex
\kloesung

\begin{soltable}{}
 A: $1,5\cdot 2^x$& 2\\
 B: $2^x+1$ & 2 \\
C: $2\cdot0,5^x-1$ & 2 \\
D: $2,0\cdot0,75^x$ & 2 \\
\end{soltable}

%!TEX root = 2024-Q2M05-Mathematik-2-HecC.tex
\kloesung

\begin{soltable}{}
\textbf{Landesabitur Hessen 2025 GK A Aufgabe 5}& \\ \hline 
 $\V{4}{3}{3}=\GeradeP{8}{3}{-3}{-4}{0}{3}{s}$& 1\\
 Für die erste Koordinate folgt $s=1$ und für die dritte Koordinate folgt $s=2$. Dies ist ein Widerspruch. P liegt nicht auf Q. & 1 \\ 
 $Q(4|3|0)$ bzw. $Q(0|3|3)$ & 1 \\ \hline
 Richtungsvektor der y-Achse ist $\V{0}{1}{0}$ & 0,5\\ 
 Es gilt $\V{-4}{0}{3}\cdot \V{0}{1}{0}=0$ & 1\\ 
 Daher verlaufen g und h senkrecht zueinander. & 0,5
\end{soltable}

%!TEX root = 2024-Q2M05-Mathematik-2-HecC.tex
\kloesung

\begin{soltable}{}
\textbf{IQB Aufgabenpool 2024 GK Teil A - AG/LA A2 Aufgabe 4}& \\
\url{https://www.iqb.hu-berlin.de/abitur/pools2024/abitur/pools2024/mathematik/mathematik grundlegend/2024_M_grundlege_11.pdf} & \\ \hline 
 $S_{1,2}=\GeradeG{-1}{4}{-2}{2}{0}{1}{x_1}{x_2}{0}$& 1\\
 $t=2$ und $S_{1,2}=M=(3|4|0)$ & 1 \\ 
 $\left|\overrightarrow{OM}\right|=\sqrt[]{3^2+4^2}=5$ & 1 \\ 
 $A=4\cdot \frac{1}{2} \cdot 5 \cdot 5 = 50$ & 2 
\end{soltable}

%!TEX root = 2024-E25-Mathematik-2-HecC.tex
\kloesung

\begin{soltable}{}
$f(6)=150\cdot e^{\num{0,25}\cdot 6}\approx 672,3$ & 1\\
Nach 6 Tagen gibt es ca. 672 Stechmücken & 1\\ \hline
$1500=f(t)$ & \\
$1500 = 150\cdot e^{\num{0,25}\cdot t}$ & 1\\
$10 = e^{\num{0,25}\cdot t}$ & \\
$\ln{10}=\num{0,25}\cdot t$ & \\
$t=4\ln{10}\approx$ 9,2103& 1\\ 
Nach 9 Tagen und ca. 5 Stunden beträgt die Population 1500 Mücken & 1 \\ \hline

$2=e^{\num{0,25}\cdot t}$ & 1\\
$t=4\ln{2}\approx$ 2,7726& 1\\ \hline
Durch Anwendung der Potenzgesetze $e^{\num{0,25}\cdot t}=\pr{e^{\num{0,25}}}^t\approx 1,284^t$. Jedes exponentielle Wachstum kann in der Form $a\cdot b^x$ oder $a\cdot e^{x\cdot\ln{b}}$ geschrieben werden & 2\\
\num{1,284} entspricht einem prozentualen Wachstum um \num{28,4}\% bzw. auf \num{128,4}\%. & 1\\ \hline
$m=\dfrac{f(6)-f(0)}{6-0}=\dfrac{150e^{1,5}-150}{6}\approx 87,0$ & 2\\
Die durchschnittliche Wachstumsgeschwindigkeit in den ersten 6 Stunden beträgt ca. 87 Mücken pro Stunde. & 1\\ \hline
$f'(t)=37,5\cdot e^{0,25t}$ & 2 \\
$f'(6)=37,5\cdot e^{1,5}\approx 168$ & 1\\
Die momentane Wachstumsgeschwindigkeit am Ende des 6. Tages beträgt ca. 168 Mücken pro Stunde. & 1 \\ \hline
$130=f(2) \Rightarrow 130=a\cdot e^{2c}$ & 0,5\\
$300=f(4) \Rightarrow 300=a\cdot e^{4c}$ & 0,5\\
$\dfrac{300}{130}=\frac{e^{4c}}{e^{2c}}$ & \\
$\frac{30}{13}=e^{2c}$ & \\
$c=\frac{\ln\pr{\frac{30}{13}}}{2}\approx 0,4181 $ & 1\\
$130=a\cdot e^{0,4181\cdot 2}$ & \\
$a=\frac{130}{e^{0,8362}}\approx 56,34$ & 1\\
$f(t)=56,34\cdot e^{0,4181t}$ &  \\
\end{soltable}

%!TEX root = 2024-E25-Mathematik-2-HecC.tex
\kloesung

\begin{soltable}{}
 $a=70; b=\left( \frac{1}{2}\right)^\frac{1}{10}≈0,9330; T(t)=70\cdot 0,9330^t$& 3\\ \hline
 $T(5)≈49,5$. Der Glühwein hat nach 5 Minuten eine Temperatur von \ang{49,5}. & 2 \\ \hline
 $1-0,9330^10≈0,647$. Nach 15 Minuten ist der Glühwein um 64,7\% abgekühlt. & 2 \\ \hline 
 $60=70\cdot 0,9330^t ~ \Leftrightarrow ~\frac{6}{7}=0.9330^t ~ \Leftrightarrow ~ t=\log_{0,9330}\left(\frac{6}{7}\right)≈2,2228$ & 2\\
 Nach 2 Minuten und 14 Sekunden hat der Glühwein Trinktemperatur erreicht. & 1\\ \hline 
 Die Gleichung $T(t)=0$ besitzt keine Lösung, da sich die Exponentialfunktion der 0 nur beliebig nah annähert, aber diese nie erreicht. Im Anwendungskontext sind aber irgendwann Temperaturen erreicht, die faktisch 0 sind. Der Ansatz $0,1=T(t)$ liefert $t≈94,5$. Nach 95 Minuten hat der Glühwein sicher 0 Grad erreicht. & 2
\end{soltable}




\end{document}

