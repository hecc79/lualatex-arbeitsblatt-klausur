%!TEX root = 2024-E25-Mathematik-2-HecC.tex
\kloesung

\begin{soltable}{}
 $a=70; b=\left( \frac{1}{2}\right)^\frac{1}{10}≈0,9330; T(t)=70\cdot 0,9330^t$& 3\\ \hline
 $T(5)≈49,5$. Der Glühwein hat nach 5 Minuten eine Temperatur von \ang{49,5}. & 2 \\ \hline
 $1-0,9330^10≈0,647$. Nach 15 Minuten ist der Glühwein um 64,7\% abgekühlt. & 2 \\ \hline 
 $60=70\cdot 0,9330^t ~ \Leftrightarrow ~\frac{6}{7}=0.9330^t ~ \Leftrightarrow ~ t=\log_{0,9330}\left(\frac{6}{7}\right)≈2,2228$ & 2\\
 Nach 2 Minuten und 14 Sekunden hat der Glühwein Trinktemperatur erreicht. & 1\\ \hline 
 Die Gleichung $T(t)=0$ besitzt keine Lösung, da sich die Exponentialfunktion der 0 nur beliebig nah annähert, aber diese nie erreicht. Im Anwendungskontext sind aber irgendwann Temperaturen erreicht, die faktisch 0 sind. Der Ansatz $0,1=T(t)$ liefert $t≈94,5$. Nach 95 Minuten hat der Glühwein sicher 0 Grad erreicht. & 2
\end{soltable}
