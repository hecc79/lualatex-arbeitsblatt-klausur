%!TEX root = 2024-E25-Mathematik-2-HecC.tex
\kaufgabe{20}

In einem Labor wird die Wirksamkeit eines neuen Mittels gegen die Ausbreitung von Stechmücken untersucht.
Bei einem ersten Versuch beschreibt die Funktion $N(t)=150\cdot e^{\num{0,25}t}$ ($t$ in Tagen) modellhaft die Population von Stechmücken. $N(t)$ ist die Anzahl der Stechmücken zum Zeitpunkt $t$. Die Beobachtung beginnt bei $t=0$.

\begin{tasks}(1)
	\task Berechnen Sie die Populationsgröße zum Zeitpunkt 6 Tage nach Beobachtungsbeginn.
	\task Berechnen Sie den Zeitpunkt, zu dem die Populationsgröße die Anzahl von 1500 Stechmücken erreicht.
	\task Berechnen Sie die Verdopplungszeit.
	\task Erklären Sie, dass die Funktion $N(t)$ auch durch die Funktionsgleichung $N(t)=150\cdot \num{1,284}^t$ beschrieben werden kann. Erläutern Sie den Wert von \num{1,284} im Sachzusammenhang.
	\task Berechnen Sie die durchschnittliche Wachstumsgeschwindigkeit während der ersten 6 Tage.
	\task Berechnen Sie die momentane Wachstumsgeschwindigkeit am Ende des 6. Tages.
	\task Von einer anderen Mückenart werden am Ende des zweiten Tages 130 Mücken und am Ende des vierten Tages 300 Mücken gezählt. Berechnen Sie eine geeignete Funktionsgleichung, die dieses Wachstum als exponentielles Wachstum beschreibt.
\end{tasks}