%!TEX root = 2024-E25-Mathematik-2-HecC.tex
\kloesung

\begin{soltable}{}
$f(6)=150\cdot e^{\num{0,25}\cdot 6}\approx 672,3$ & 1\\
Nach 6 Tagen gibt es ca. 672 Stechmücken & 1\\ \hline
$1500=f(t)$ & \\
$1500 = 150\cdot e^{\num{0,25}\cdot t}$ & 1\\
$10 = e^{\num{0,25}\cdot t}$ & \\
$\ln{10}=\num{0,25}\cdot t$ & \\
$t=4\ln{10}\approx$ 9,2103& 1\\ 
Nach 9 Tagen und ca. 5 Stunden beträgt die Population 1500 Mücken & 1 \\ \hline

$2=e^{\num{0,25}\cdot t}$ & 1\\
$t=4\ln{2}\approx$ 2,7726& 1\\ \hline
Durch Anwendung der Potenzgesetze $e^{\num{0,25}\cdot t}=\pr{e^{\num{0,25}}}^t\approx 1,284^t$. Jedes exponentielle Wachstum kann in der Form $a\cdot b^x$ oder $a\cdot e^{x\cdot\ln{b}}$ geschrieben werden & 2\\
\num{1,284} entspricht einem prozentualen Wachstum um \num{28,4}\% bzw. auf \num{128,4}\%. & 1\\ \hline
$m=\dfrac{f(6)-f(0)}{6-0}=\dfrac{150e^{1,5}-150}{6}\approx 87,0$ & 2\\
Die durchschnittliche Wachstumsgeschwindigkeit in den ersten 6 Stunden beträgt ca. 87 Mücken pro Stunde. & 1\\ \hline
$f'(t)=37,5\cdot e^{0,25t}$ & 2 \\
$f'(6)=37,5\cdot e^{1,5}\approx 168$ & 1\\
Die momentane Wachstumsgeschwindigkeit am Ende des 6. Tages beträgt ca. 168 Mücken pro Stunde. & 1 \\ \hline
$130=f(2) \Rightarrow 130=a\cdot e^{2c}$ & 0,5\\
$300=f(4) \Rightarrow 300=a\cdot e^{4c}$ & 0,5\\
$\dfrac{300}{130}=\frac{e^{4c}}{e^{2c}}$ & \\
$\frac{30}{13}=e^{2c}$ & \\
$c=\frac{\ln\pr{\frac{30}{13}}}{2}\approx 0,4181 $ & 1\\
$130=a\cdot e^{0,4181\cdot 2}$ & \\
$a=\frac{130}{e^{0,8362}}\approx 56,34$ & 1\\
$f(t)=56,34\cdot e^{0,4181t}$ &  \\
\end{soltable}
