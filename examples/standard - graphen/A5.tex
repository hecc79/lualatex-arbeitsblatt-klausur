%!TEX root = 2024-E25-Mathematik-2-HecC.tex
\kaufgabe{12}

Der Abiturjahrgang 2025 hat vor Weihnachten heißen Glühwein auf dem Weihnachtsmarkt verkauft, um die Abikasse aufzubessern. Der Glühwein wird auf \ang{70}C erhitzt. Wenn der Glühwein in eine Tasse gefüllt wird, kühlt dieser bei einer Umgebungstemperatur von \ang{0}C innerhalb von 10 Minuten um die Hälfte ab.

\begin{tasks}(1)
	\task Leiten Sie eine Funktionsgleichung $T(t)$ her, die die oben beschriebene Abkühlung von Glühwein beschreibt. $T(t)$ gibt hierbei die Temperatur in Grad Celsius und $t$ die Zeit in Minuten an. Runden Sie auf vier Nachkommastellen. 

	[Kontrollergebnis: $T(t)=70\cdot e^{-0,0693t}$]
	\task Bestimmen Sie die Temperatur des Glühweins nach 5 Minuten.
	\task Berechnen Sie, um wie viel Prozent der Glühwein nach 15 Minuten abgekühlt ist.
	\task Bei einer Temperatur von \ang{60}C lässt sich der Glühwein angenehm trinken. Berechnen Sie, wie lange ein Kunde warten muss, bis diese Temperatur erreicht ist.
	\task Diskutieren Sie, wann der Glühwein eine Temperatur von \ang{0}C erreicht hat.
\end{tasks}